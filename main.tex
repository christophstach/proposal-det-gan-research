\documentclass[conference,onecolumn,compsoc]{IEEEtran}


\usepackage[onehalfspacing]{setspace}
\usepackage[stretch=10]{microtype}

\usepackage[pdftex]{graphicx}

\usepackage{amsmath}

\usepackage{algorithmic}
\usepackage{array}

\usepackage[caption=false,font=normalsize,labelfont=sf,textfont=sf]{subfig}

\usepackage{dblfloatfix}
\usepackage[nomarkers]{endfloat}

\usepackage{url}
\usepackage{hyperref}
\usepackage{csquotes}

\usepackage[english]{babel}
\usepackage{blindtext}

\usepackage[style=ieee]{biblatex}
\addbibresource{references/papers.bib}


\begin{document}


\title{Proposal: Image Generation with GANs}

\author{
  \IEEEauthorblockN{Christoph Stach}
  \IEEEauthorblockA{
    Hochschule für Technik und Wirtschaft Berlin\\
    Fachbereich 4 - Angewandte Informatik\\
    s0555912@htw-berlin.de
  }
}
\maketitle

% As a general rule, do not put math, special symbols or citations
% in the abstract or keywords.
\begin{abstract}

\noindent
In this project the goal is to use a generative 

\end{abstract}


\section{Motivation}

\noindent
\blindtext


\section{Objective}

\noindent
\blindtext


\section{Procedure}

\noindent
\blindtext

\section{Requirements and constraints}

\blindtext

\section{Timetable}

\noindent
\blindtext \cite{yazıcı2019unusual}

\section{Expected Results}

\noindent
\blindtext


\section{Preliminary  Structure}

\noindent
\begin{enumerate}
  \item Abstract
  \item Introduction
  \item Basics \cite{goodfellow2014generative}
  \item Technologies
  \begin{enumerate}
    \item WGAN-GP \cite{arjovsky2017wasserstein,gulrajani2017improved}
    \item MSG-GAN \cite{karnewar2020msggan}
    \item Adam \cite{kingma2017adam}
    \item Relatavitic Discriminator Loss \cite{jolicoeurmartineau2018relativistic}
    \item Exponential Moving Average \cite{yazıcı2019unusual}
    \item Spectral Normalization \cite{miyato2018spectral}
  \end{enumerate}
  \item Implementation
  \item Evaluation
  \item Conclusion and future applications
\end{enumerate}


\newpage


\printbibliography


\end{document}
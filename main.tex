\documentclass[conference,onecolumn,compsoc]{IEEEtran}


\usepackage[onehalfspacing]{setspace}
\usepackage[stretch=10]{microtype}

\usepackage[pdftex]{graphicx}

\usepackage{amsmath}

\usepackage{algorithmic}
\usepackage{array}

\usepackage[caption=false,font=normalsize,labelfont=sf,textfont=sf]{subfig}

\usepackage{dblfloatfix}
\usepackage[nomarkers]{endfloat}

\usepackage{url}
\usepackage{hyperref}
\usepackage{csquotes}

\usepackage[english]{babel}
\usepackage{blindtext}

\usepackage[style=ieee]{biblatex}
\addbibresource{references/papers.bib}


\begin{document}


\title{Proposal: Image Generation with GANs}

\author{
  \IEEEauthorblockN{Christoph Stach}
  \IEEEauthorblockA{
    Hochschule für Technik und Wirtschaft Berlin\\
    Fachbereich 4 - Angewandte Informatik\\
    s0555912@htw-berlin.de
  }
}
\maketitle

% As a general rule, do not put math, special symbols or citations
% in the abstract or keywords.
\begin{abstract}

\noindent
The research project of the in area of computer science will be about image generation with Generative Adversial Networks. The goal is to examinate GANs and explore their capatiblity of generating new, unseen images of arbitary datasets. In this project esspecially foto realistic images are used but the proposed application is also able to generate images of a different format. Generative Adversial Networks is a research area of deep learning, Where to two (or more) Neural Networks are competing with each other. In a standard GAN architecture a Generator generates images from a random number vector. A discrimator is deciding if the image comes from the Generator or the true Dataset. The Generator continuously tries to fool the Discriminator and therefore creates Images of increasing quality.

\end{abstract}


\section{Motivation}

\noindent
The first GANs implementations used Dense layers to generate new data \cite{goodfellow2014generative}. Follow-up projects, like DCGAN \cite{radford2016unsupervised} used Convulional layers, which are better suited for computer vision and image tasks. Still, it was very hard to train networks with the capability of generating larger images. Training offten suffers from problems like Mode-Collapse. Also GANs require a datasets of large size and sufficient compute power. For these reasons GANs are an interesting research area, which highly motivates to explore it.

\section{Objective}

\noindent
The goal of the project is to get more familary with Generative Adversial Networks, it functionalities, features but also its pitfalls and problems. Only by understanding GANs better it is possible to improve them in future projects.

https://github.com/bchao1/Anime-Face-Dataset


\section{Procedure}

\noindent
\blindtext

\section{Requirements and constraints}

\blindtext

\section{Timetable}

\noindent
\blindtext \cite{yazıcı2019unusual}

\section{Expected Results}

\noindent
\blindtext


\section{Preliminary  Structure}

\noindent
\begin{enumerate}
  \item Abstract
  \item Introduction
  \item Basics \cite{goodfellow2014generative}
  \item Technologies
  \begin{enumerate}
    \item WGAN-GP \cite{arjovsky2017wasserstein,gulrajani2017improved}
    \item MSG-GAN \cite{karnewar2020msggan}
    \item Adam \cite{kingma2017adam}
    \item Relatavitic Discriminator Loss \cite{jolicoeurmartineau2018relativistic}
    \item Exponential Moving Average \cite{yazıcı2019unusual}
    \item Spectral Normalization \cite{miyato2018spectral}
  \end{enumerate}
  \item Implementation
  \item Evaluation
  \item Conclusion and future applications
\end{enumerate}


\newpage


\printbibliography


\end{document}
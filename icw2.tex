%\documentclass[conference,onecolumn,compsoc]{IEEEtran}
\documentclass[]{article}


\usepackage[onehalfspacing]{setspace}
\usepackage[stretch=10]{microtype}

\usepackage[pdftex]{graphicx}

\usepackage{amsmath}

\usepackage{algorithmic}
\usepackage{array}

\usepackage[caption=false,font=normalsize,labelfont=sf,textfont=sf]{subfig}

\usepackage{dblfloatfix}
\usepackage[nomarkers]{endfloat}

\usepackage{url}
\usepackage{hyperref}
\usepackage{csquotes}

\usepackage[english]{babel}
\usepackage{blindtext}

\usepackage[style=ieee]{biblatex}
\addbibresource{references/papers.bib}


\begin{document}


\title{\textbf{Exposé: Improving the implementation of a GAN-Architecture}}

\author{
  Christoph Stach\\
  Hochschule für Technik und Wirtschaft Berlin\\
  Fachbereich 4 - Angewandte Informatik\\
  s0555912@htw-berlin.de
}
\maketitle

% As a general rule, do not put math, special symbols or citations
% in the abstract or keywords.
\begin{abstract}

\noindent
The research project in the field of computer science will be about image generation with Generative Adversarial Networks. The goal is to examine GANs and explore their capability of generating new unseen images of arbitrary datasets. This project, especially photorealistic images, are used, but the proposed application can also generate a different format. Generative Adversarial Networks is a research area of deep learning, where two (or more) Neural Networks are competing. In a standard GAN architecture, a Generator generates images from a random number vector. A discriminator decides if the image comes from the Generator or the real Dataset. The Generator continuously tries to fool the Discriminator and therefore creates images of increasing quality.

\end{abstract}


\section{Motivation}

\noindent


\section{Objective}

\noindent


\section{Procedure}

\noindent


\section{Requirements and constraints}

\noindent


\section{Expected Results}

\noindent

\newpage

\section{Preliminary Structure}

\noindent
\begin{enumerate}
  \item Abstract
  \item Introduction
  \item Objective
  \item Fundanmentals
  \begin{enumerate}
      \item MSG-GAN \cite{lee2020maskgan}
      \item EMA \cite{yazıcı2019ema}
      \item Spectral Normalization \cite{miyato2018spectral}
      \item Relative Discriminator Loss \cite{jolicoeurmartineau2018rahinge}
  \end{enumerate}
  \item Methodology
  \begin{enumerate}
    \item Technologies
    \begin{enumerate}
        \item PyTorch
        \item Determined.ai
    \end{enumerate}
    \item Datasets
    \item Implementation
    \item Evaluation
  \end{enumerate}
  \item Conclusion
  \item Future work and improvements
\end{enumerate}

\printbibliography


\end{document}